\documentclass[letterpaper,10pt,titlepage]{article}

\usepackage{graphicx}                                        
\usepackage{amssymb}                                         
\usepackage{amsmath}                                         
\usepackage{amsthm}                                          

\usepackage{alltt}                                           
\usepackage{float}
\usepackage{color}
\usepackage{url}

\usepackage{balance}
%\usepackage[TABBOTCAP, tight]{subfloat}
\usepackage{enumitem}
\usepackage{pstricks, pst-node}

\usepackage{geometry}
\geometry{textheight=8.5in, textwidth=6in}

%random comment

\include{pygments.tex}

\newcommand{\cred}[1]{{\color{red}#1}}
\newcommand{\cblue}[1]{{\color{blue}#1}}

\usepackage{hyperref}
\usepackage{geometry}

\def\name{Neale Ratzlaff}
\title{Computer Architecture Assignment 1}
\author{Neale Ratzlaff}
\date{30 September 2015}

%pull in the necessary preamble matter for pygments output
\input{pygments.tex}


%% The following metadata will show up in the PDF properties
\hypersetup{
  colorlinks = true,
  urlcolor = black,
  pdfauthor = {\name},
  pdfkeywords = {cs311 ``operating systems'' files filesystem I/O},
  pdftitle = {CS 311 Project 1: UNIX File I/O},
  pdfsubject = {CS 311 Project 1},
  pdfpagemode = UseNone
}

\begin{document}
\maketitle
\pagebreak

%input the pygmentized output of foo.c, using a (hopefully) unique name
%this file only exists at compile time. Feel free to change that.
\section{Difference between architecture and organization}
    \newline
    \newline
    Organization is what most people think of when they think of architecture,
    when organization is mostly defined by the ISA. Organization is the physical layout of the 
    hardware. The layout of the memory stack, how the ICs are alligned onboard are all part 
    of the organization of the computer. This is important as data transfer speeds, power
    consumption, and system functionality are all affected by the organization. 
    \\*
    \\*
    Architecture is essentially the computer's instruction set. The set of rules that define how 
    the processor will behave, which in turn necessitates organizational features. The architecture
    may also include the microarchitecture of the processor, or how the processor will implement
    each instruction in the ISA (a black box to users). 


\pagebreak

\section{Endianness}
    \\*
    \\*
    Endianness is the concept of ordering with respect to bits in the system 
    There are two types of endianness in modern computers, big-endian and little-endian 
    In a big-endian system, the smallest memory address stores the most significant byte of a memory word 
    Little-endian systems store the least significant byte in the largest memory address. 
    Intel's x86 architecture uses little-endian representations. Little-endian systems 
    are also common with microprocessors due to Intel's influence. Big-endian 
    systems are most commonly found in computer networking applications. 
    There is also another type called a mixed-endian system. A mixed system will have 
    a different endianness for 16bit words vs. 32 bit words. Bi-endian processors 
    can operate in little-endian or big-endian mode. 
    \\*
    \\*
    
\pagebreak

\section{Floating point format (IEEE)}

    \\*
    \\*
    Single precision format is known as binary32. Binary32 consists of 
    \begin{itemize}
        \item 1 bit sign bit
        \item 8 bit exponent bit
        \item 24 bit significand component (23 explicitly stored)
    \end{itemize}
    \\*
    Double precision format is called binary64:
    \begin{itemize}
        \item 1 bit sign bit
        \item 11 bit exponent bit
        \item 24 53 significand component (52 explicitly stored)
    \end{itemize}

\pagebreak

\section{Memory heirarchy}

    \\*
    \\*
    The memory heirarchy of the computer refers to the arrangement of the memory locations 
    of the system. The smallest and fastest memory cache is the L1 cache. L1 is further split 
    into two levels of instruction and data cache, each 128 KiB. The instruction cache contains a block 
    of opcode that has been fetched from memory. By temporal locality it can be assumed that code
    that instructions that have been executed recently are more likely to be used again, so having 
    rencent instructions in the L1 cache allows for increased loop execution. The L1 data cache operates
    on the same principal, only it stores program data instead of instructions. The L1 data transfer speed
    reaches 700 GiB/s. 
    \\
    The next cache in the heirarchy is the L2 cache. The L2 is a unified data and instruction cache 1 MiB 
    in size. L2 is a slower and farther away cache than L1, best access speed is 200 GiB/s. The L2 cache 
    normally feeds data into the L1 cache. 
    \\
    The last level found in generally all systems is the L3 cache. This is the largest cache still at 6MiB. 
    L3 is generally a shared cache between all the cores of the processor. L1 and L2 usually are dedicated to each 
    core to improve performance. While L3 is a large shared space used as the big-brother backup to the smaller
    L1 and L2 caches. Access speed of the L3 cache is around 100 GB/s. 
    \\
    Some processors have an L4 cache that is the largest at 128 MiB, and acts as just one more level of cache 
    memory. Beyond this is only main memory.


\end{document}
